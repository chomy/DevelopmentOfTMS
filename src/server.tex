\chapter*{タイルサーバの作成}

\section*{タイルの作成}
では、タイルを作成しましょう。
元になる、測地系といった地理空間情報が付いた画像があれば簡単です。
例えば、world.tiffという、GeoTiffファイルがあったとしましょう。
GeoTiffファイルは、地理空間情報のタグが付加されたTiff画像です。
地理空間情報が付いたフォーマットにShape等もありますが、21世紀なのでGeoTiffで良いでしょう。
タイルの作成には、GDALに入っているgdal2tiles.pyを使います。
GDALは、''ぐーだる''とか''ぐだーる''と呼ばれている、Geoな世界のスイスアーミーナイフ的な
存在で、本来はC/C++のライブラリなのですが、付属するプログラムが優秀でツール群と誤解されているものです。
後で、プログラミングライブラリとして使用しますが、タイルの生成には、GDALをツール群として使います。GDALのインストールは面倒なイメージがあるので、LinuxのパッケージやOSGeoLiveといったディストロを使うと良いでしょう。
特にOSGeoLiveはGeoな事をやる時に必要なツールはほとんど入っているので、特におすすめです。

タイルの作り方ですが、gdal2tiles.pyにGeoTiffファイル名を指定するだけです。
生成するタイルのズームレベルの範囲を指定することもできます。
たとえば、map.tiffから、ズームレベル0から8までのタイルを生成するには、以下のコマンドを実行します。

%\begin{center}
\texttt{\$ gdal2tiles.py -z 0-8 map.tiff}
%\end{center}

タイルの生成には、大量の画像を生成するためものすごーく時間がかかります。
ちなみにズームレベル0から8までのタイルの総数は87,381枚です。


