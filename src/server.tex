\chapter*{タイルサーバの作成}

\section*{Web Serverの準備}
TMSは、タイルをHTTPプロトコルで配布します。
ですので、ネットワークで到達可能な場所に
Webサーバを準備しなければなりません。
例えば、AWSのEC2やさくらVPS、IDCクラウドなどで
インスタンスを立てれば良いでしょう。
%
OSはLinuxであればディストリビューションは問いませんが
私はDebianを使っているので、Debianを使用した例をご紹介します。
Ubuntuでもほとんどは変わらないはずです。
BSD派の人は、試してはいませんが、FreeBSDでもOpenBSDでもNetBSDでもOKなはずです。

一般に公開せずに、ただやってみるだけであれば、Virtualbox等の
仮想PCでも良いでしょう。
Virtualboxであれば無料でダウンロードすることができます。

さてWeb Serverは、nginxを使います。apacheが慣れているのであれば
apacheでも良いでしょう。aptでインストールしましょう。
(正直、HTTPで画像が配布できれば良いので、何でもよいです)

\section*{タイルの作成}
では、タイルを作成しましょう。
タイルの作成には、GDALに入っているgdal2tiles.pyを使います。
GDALは、''ぐーだる''とか''ぐだーる''と呼ばれている、Geoな世界のスイスアーミーナイフ的な
存在で、本来はC/C++のライブラリなのですが、付属するプログラムが優秀でツール群と誤解されているものです。
後で、プログラミングライブラリとして使用しますが、タイルの生成には、GDALをツール群として使います。GDALのインストールは面倒なイメージがあるので、LinuxのパッケージやOSGeoLiveといったディストロを使うと良いでしょう。
特にOSGeoLiveはGeoな事をやる時に必要なツールはほとんど入っているので、特におすすめです。

タイルの作り方ですが、gdal2tiles.pyにGeoTiffファイル名を指定するだけです。
生成するタイルのズームレベルの範囲を指定することもできます。
たとえば、map.tiffから、ズームレベル0から8までのタイルを、/var/www/html/に生成するには、以下のコマンドを実行します。

%\begin{center}
\texttt{\$ gdal2tiles.py -z 0-8 map.tiff /var/www/html/}
%\end{center}

タイルの生成には、大量の画像を生成するためものすごーく時間がかかります。
ちなみにズームレベル0から8までのタイルの総数は87,381枚です。

Webサーバのホスト名が、tms.example.orgだったとすると、
http://tms.example.org/{z}/{x}/{y}.pngというURLで
タイルを取得することができます。
