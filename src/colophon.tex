\section*{編集後記}

「タイルマップサービスを作ってみた」をお送りしました。

今回は、4月に技術書典2で配布したものに、
カラーテーブルの作成、Leafletによる描画等を大幅加筆しました。
技術書典3もサークル参加が決まっているので、OpenLayers4による描画や
要求されたタイルをダイナミックに生成するタイルサーバ等
の実装についても加筆できればなーと思っています。
もちろん加筆後も、PDFでGitHubでダウンロード可能にします。

この仕事は、5年ほど前からやっているaerial-projectという
プロジェクトの一部で、ずいぶん長くやっています。
特に今の時季は、台風があるので、JTWCの進路予想と
海水面温度を重ねて、勢力の推定とかできたりします。

とはいえ、そろそろ飽きたので違う事をやろうかとも思ってます。
例えば、SDR(Software Defined Radio)という、ソフトウェアで
無線をごにょごにょするデバイスで遊んでいるので、そのネタとか。
(だがしかし、このネタではUNIX島では売れないんだよなー)
あと、C++14で導入された右辺値参照をアセンブラで検証するネタとか。
(怖い人がいっぱいいる...)

まぁなんにせよ何か書きます。
新刊になるか、このネタの続きになるかは未定ですが。


最後に、この同人誌はDebian/GNU Linux、\TeX Live2016、
psutils、git、GNU Make、vimといった、オープンソースソフトウェアを使って作成されました。
また日本語のフォントはIPAexフォントをPDFに埋め込んでいます。
このような有益なソフトウェアを開発、維持、管理していただいているすべての皆様に感謝します。
また、このページまでたどり着いてくれた読者の方(おそらくあなただけです)に感謝します。
ありがとうございました。

\begin{flushright}
2017年8月 Keisuke Nakao (@jm6xxu) 
\end{flushright}

\section*{参考文献}
         [1] ``Tile Map Service in Geoide'', \texttt{http://geoikia.idgis.eu/wiki-english/index.php}
       
       [2] ``Slippy map tilenames'', \texttt{http://wiki.openstreetmap.org/wiki/Slippy\_map\_tilenames}
       
[3] GCOM-W1 データ提供サービス \texttt{https://gcom-w1.jaxa.jp/auth.html}



\clearpage
\mbox{}
\vspace{55em}\\
この作品はクリエイティブ・コモンズ・ライセンス 表示 - 継承 2.1 日本 の下に提供されています。このライセンスのコピーを見るためには、http://creativecommons.org/licenses/by-sa/2.1/jp/ をご覧ください。
