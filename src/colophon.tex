\section*{編集後記}

「タイルマップサービスを作ってみた」をお送りしました。
が、時間がなくて内容がペラペラで申し訳ありません。
夏コミもこのネタでいく予定ですので、それまでに加筆します。

書くべき内容として、カラーテーブルの作成方法やLeafletやOpenLayersを使った
ブラウザ表示アプリ、Webサーバの設定等があります。
また、ブラウザから要求された必要なタイルだけダイナミックに生成するサービスも
できているので、それについても書きたかった...orz

このような事になった言い訳としては、結婚して自由になる時間が減った、というか
リア充過ぎて薄い本のネタなど作っている暇がないというのが実情です。
また、研究者からITセキュリティエンジニアに転職して、家に返ってきてまでPC触りたくない
というのもあります。
と、奥様やら仕事やらに責任転嫁していますが、創造性が失われてきたという現実も認識しています。
これは結構ヤバい。

この厳しい現実に贖いつつ、夏コミまでには加筆修正しますので、冒頭に書いたGitHubリポジトリで
加筆後のコンテンツを入手することができます。
というか、お代は印刷代にすらならない事もあり、これまでに書いたものも公開していますのでよろしければご覧ください。

最後に、この同人誌はDebian/GNU Linux、\TeX Live2016、
psutils、git、GNU Make、vimといった、オープンソースソフトウェアを使って作成されました。
また日本語のフォントはIPAexフォントをPDFに埋め込んでいます。
このような有益なソフトウェアを開発、維持、管理していただいているすべての皆様に感謝します。
また、このページまでたどり着いてくれた読者の方(おそらくあなただけです)に感謝します。
ありがとうございました。

\begin{flushright}
2017年4月 Keisuke Nakao (@jm6xxu) 
\end{flushright}

\clearpage
\mbox{}
%\clearpage
%\mbox{}
\vspace{55em}\\
この作品はクリエイティブ・コモンズ・ライセンス 表示 - 継承 2.1 日本 の下に提供されています。このライセンスのコピーを見るためには、http://creativecommons.org/licenses/by-sa/2.1/jp/ をご覧ください。
