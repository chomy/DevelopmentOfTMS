\chapter*{はじめに}

さすがにC92でこんな本に興味を示してくれた人が、Google Mapを知らないということは
ないでしょう。この本はGoogle Mapのようにぐりぐり動く、いわゆるSlippy Mapを使った
Webサービスを開発する方法が書かれています。
すでに地形図や道路地図は、Google Mapだけでなく、地理院地図やOpenStreetMapがあります。
今回は、この地理院地図やOpenStreetMapを基本図として、その上に
水循環変動観測衛星「しずく」(GCOM-W)という地球観測衛星が観測した
海水面温度(SST)のデータを重ねた情報を提供するサービスを開発します。
このデータは、JAXAに利用申請をすると、無料で使用することができます。

SSTは、数値データとして提供されますので、まず画像データとして可視化し、その画像に位置情報を付加し、タイルマップを
生成していきます。
気象データが相手ですから、定期的にデータをアップデートする必要があります。
そこで、QGISのようなGUIのプログラムではなく、自動化を見据えて、コマンドラインプログラムだけで作成していきます。

この本で作成するWebアプリは、Debian/GNU Linux上で動作する事を前提にしており、Linuxの基本操作ができる人を対象にしています。
また、Webサービスを開発するだけの基礎的な素養がある事を前提にしています。
適宜解説に努めますが、十中八九説明が足りないので、必要に応じてリファレンス等を参照してください。

この本に出てくるソースコードは、GitHubで公開しています。また電子版もGitHubから入手できます。間違い等ありましたら、Issueを投げていただければ対応できると思います。
リポジトリのURLは以下の通りです。\\
\texttt{https://github.com/chomy/DevelopmentOfTMS}


