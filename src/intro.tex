\chapter*{はじめに}

さすがに技術書典でこんな本に興味を示してくれた人が、Google Mapを知らないということは
ないでしょう。この本はGoogle Mapのようにぐりぐり動く、いわゆるSlippy Mapを使った
Webサービスを開発する方法が書かれています。
すでに地形図や道路地図は、Google Mapだけでなく、地理院地図やOpenStreetMapがあります。
今回は、この地理院地図やOpenStreetMapを基本図として、その上にGSMaPと呼ばれる
降水量データを重ねた情報を提供するサービスを開発します。

GSMaPは、JAXA、NOAA、EUMETSATが運用している気象衛星のデータを準リアルタイム(といっても4時間前のデータですが)の数値データが1時間毎に提供されています。
JAXAに利用申請をすると、無料で使用することができます。
この数値データをまず画像データとして可視化し、その画像に位置情報を付加し、タイルマップを
生成していきます。
気象データが相手ですから、毎時自動的にデータをアップデートする必要があります。
そこで、QGISのようなGUIのプログラムではなく、できるだけコマンドラインだけで作成していきます。

この本で作成するWebアプリは、Debian/GNU Linux上で動作する事を前提にしています。
ですので、Linuxの基本的な操作ができる人を対象にしています。
また、Webサービスを開発するだけの基礎的な素養がある事を前提にしています。
nginxの設定ファイルやPythonやシェルスクリプト、C++等のソースコードが出てきます。
適宜解説に努めますがが、十中八九説明が足りないので、必要に応じてリファレンス等を参照すると
良いでしょう。

この本に出てくるソースコードは、GitHubで公開しています。また電子版もGitHubから入手できます。間違い等ありましたら、ISSUEを投げていただければ対応できると思います。
リポジトリのURLは以下の通りです。\\
\texttt{https://github.com/chomy/DevelopmentOfTMS}


